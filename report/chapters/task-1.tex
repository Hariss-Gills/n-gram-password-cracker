For this task, the most challenging part was definitely generating the set of shortlex strings. After remembering the definition from Modeling and Problem Solving for Computing course, the shortlex order is simply a total ordering of the lexicographical order at a certain length \cite{wikipediaShortlexOrder}. To compute the lexicographical order of a set at a certain length, find the order on the Cartesian product of the ordered set - in this case ASCII lowercase and digits \cite{math24}. While writing this report, I came across the use of generators in Python to save some memory when returning so many strings \cite{python}.

As for the algorithm itself, the output list was first initialized with None values. This was done to ensure that the indices match up i.e The input list hash would have the same index as the output plain text password. The function keeps running until it finds all of the passwords hence the stop condition is when the output list has no \verb|None| values. Instead of iterating through the list of hashes which would generate the same strings a lot of times, the loop uses the list of shortlex strings. Currently, a limitation is when the input list contains non-unique hashes.