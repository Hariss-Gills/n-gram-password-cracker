The second task felt relatively straightforward. The first step was to hash all of the values in the dictionary text file. A dictionary was used with the keys are the hashed passwords since it made the cracking algorithm more readable. With a list the indices would have to match the line and in turn the plain text password which would be too many read operations. Next, we check if the password was already hashed. if so, we skip hashing it. This check actually makes the speed slower, but as per the assessments' assumption "you should assume that computing hashes is more expensive than searching lists." Hence this check was kept. An improvement for speed could be to hash the file in chunks instead of the whole thing at once.

The function \verb|dictionary_cracking| uses the same indices trick as before to match the input list. This time the hash list is looped, and is enumerated to get the index. Also, in this algorithm, it's possible that the returned list may have some \verb|None| values.